\documentclass[14pt,a4paper,oneside]{report}
\usepackage[utf8]{inputenc}
\usepackage[english,russian]{babel}
\linespread{1.1}
\usepackage{indentfirst}

\pagestyle{headings}
\makeatletter
\renewcommand{\@evenhead}{\hfil \thepage\ \hfil}
\renewcommand{\@oddhead}{\hfil \thepage\ \hfil}
\renewcommand\thesection{\@arabic\c@section.}
\renewcommand\thesubsection{\thesection\@arabic\c@subsection.}
\renewcommand\thesubsubsection{\thesubsection\@arabic\c@subsubsection.}
\renewcommand\section{\@startsection {section}{1}{\z@}%
                                   {-3.5ex \@plus -1ex \@minus -.2ex}%
                                   {2.3ex \@plus.2ex}%
                                   {\normalfont\normalsize}}
\renewcommand\subsection{\@startsection{subsection}{2}{\z@}%
                                     {-3.25ex\@plus -1ex \@minus -.2ex}%
                                     {1.5ex \@plus .2ex}%
                                     {\normalfont\normalsize}}
\renewcommand\subsubsection{\@startsection{subsubsection}{2}{\z@}%
                                     {-3.25ex\@plus -1ex \@minus -.2ex}%
                                     {1.5ex \@plus .2ex}%
                                     {\normalfont\normalsize}}
\makeatother

\usepackage{xspace}
\newcommand{\vrio}{\texttt{vrio}\xspace}

\begin{document}
\centerline{КОМПЛЕКС ПРОГРАММ}
\centerline{<<ВИРТУАЛЬНАЯ КОММУНИКАЦИОННАЯ СРЕДА VRIO>>}
\vskip 3mm
\centerline{Технические требования}
\vskip 4mm

%\setcounter{page}{1}

\section{ВВЕДЕНИЕ}
Настоящие технические требования распространяются на разработку комплекса программ <<Виртуальная коммуникационная среда VRIO>> (далее, \vrio).

\section{ОСНОВАНИЕ ДЛЯ РАЗРАБОТКИ}
Разработка проводится на основании планов ОАВМ НИИСИ РАН.

\section{НАЗНАЧЕНИЕ РАЗРАБОТКИ}
Основное назначение программы \vrio заключается в предоставлении пользователю средства для имитационного моделирования работы коммуникационной среды RapidIO; комплекс программ должен
обеспечивать:
\begin{itemize}
\item запуск нескольких экземпляров эмулятора \texttt{qemu}, имеющих в своём составе модель
контроллера RapidIO СБИС 1890ВМ8Я и их объединение в единую коммуникационную среду RapidIO;
\item запуск на экземплярах эмулятора оригинального (то есть, работающих на аппаратных
реализациях 1890ВМ8Я и сети RapidIO) встроенного ПО (PMON2000) и ПО обработки сигналов
на базе БЦОС (под управлением ядра linux).
\end{itemize}

\section{ИСХОДНЫЕ ДАННЫЕ, ЦЕЛЬ И РЕШАЕМЫЕ ЗАДАЧИ}
\subsection{Исходные данные}

Исходными данными для разработки являются:
\begin{itemize}
\item описание функционального и эксплуатационного назначения;
\item документы <<RapidIO Interconnect Specification, Rev. 1.3, 06/2005>>, а именно:
  \begin{itemize}
  \item Part 1: Input/Output Logical Specification;
  \item Part 2: Message Passing Logical Specification;
  \item Part 3: Common Transport;
  \item Part 4: Physical Layer 8/16 LP-LVDS Specification;
  \item Part 6: 1x/4x LP-Serial Physical Layer Specification.
  \end{itemize}

\item документ <<Контроллер RapidIO СБИС 1890ВМ8Я>>;
\item частино реализованный комплекс программ \vrio в ветке \texttt{antony.20171107.vrio} в git-репозитории \texttt{ssh://git@duna.niisi.ru/qemu/qemu.git}.
\item исходные тексты драйверов оконечного устройства RapidIO СБИС 1890ВМ8Я для ПО PMON2000 и ядра linux;
\item исходные тексты оконечного устройства \vrio для qemu-v2.8.0-rc2;
\item реализация коммутатора vrio \texttt{vrio\_switch.py};
\item описание форматов пакетов vrio.
\end{itemize}

\subsection{Цель работы}

Целью работы является комплекс программ \vrio: эмулятор СБИС 1890ВМ8Я,
в состав которого входит модель оконечного устройства RapidIO;
модель коммутатора RapidIO; методические указания по использованию
комплекса программ.

\section{ТЕХНИЧЕСКИЕ ТРЕБОВАНИЯ}

\subsection{Принцип работы \vrio}

Виртуальная коммуникационная среда RapidIO (\vrio) обеспечивает
передачу данных между виртуальными оконечными устройствами \vrio
пакетов, с полями, аналогичными полям пакетов реальной коммуникационной
сети RapidIO через TCP-соединения.

Для передачи пакетов используются коммутаторы \vrio и каналы \vrio.
Каналы \vrio используются для соединения портов оконечных портов
и коммутаторов \vrio.

\subsubsection{Реализация оконечного устройства \vrio}
Оконечное устройство \vrio --- программный модуль для эмулятора qemu.
С точки зрения ПО, исполняющегося на эмуляторе, оконечное устройство
\vrio является блоком управляющих регистров в адресном пространстве
процессора, а также линия запроса на прерывание.
С точки зрения инструментальной ЭВМ, на которой исполняется
qemu, порт \vrio представляется как TCP-сервер, который ожидает
подключения от канала \vrio.

Оконечное устройство на базе программной модели контроллера RapidIO
СБИС 1890ВМ6Я должно поддерживать работу с операциями MAINTENANCE и MESSAGE.

%Оконечное устройство \vrio
%Если подключения нет, то в соответствующем
%статусном регистре Port n Error and Status CSR блока физического уровня
%оконечного устройства значение поля Port Error равно 1,
%а после подключения канала \vrio Port OK равно 1.

%Оконечное устройство отвечает на запросы MAINTENANCE READ.


\subsubsection{Реализация канала \vrio}

Реализация канала \vrio производится самыми простыми средствами,
вот, например, реализация на shell при помощи netcat:
\begin{verbatim}
PORT1=2021
PORT2=3001

FIFO1=01
FIFO2=10

rm -f $FIFO1 $FIFO2
mkfifo $FIFO1
mkfifo $FIFO2
cat < $FIFO2 > $FIFO1 &
nc6 --disable-nagle localhost $PORT1 < $FIFO1 |
    nc6 --disable-nagle localhost $PORT2 > $FIFO2
\end{verbatim}

\subsubsection{Реализация коммутатора \vrio}

Коммутатор \vrio уже реализован: \verb#vrio_switch.py#, вопрос о его доработке
или замене решается по итогам реализации поддержки операций MESSAGE.

\subsection{Требования к информационной и программной совместимости}

\subsubsection{Комплекс программ \vrio должен работать под управлением ОС Debian Linux 9.0 или новее.}

\subsubsection{Файлы с исходными текстами комплекса программ \vrio должны в git-репозитории \texttt{ssh://git@duna.niisi.ru/qemu/qemu.git}.}

\subsubsection{Исходными тексты модуля для qemu должны быть оформлены
в соответствиями требованиям оформления кода qemu; в исходных текстах
модуля qemu должны использоваться макроопределения для контроллера
RapidIO 1890ВМ6Я из состава исходных текстов ядра linux.}

\subsection{Требования к программной документации}
\subsubsection{Разрабатываемые программные модули должны быть самодокументированы, то есть тексты программ должны содержать все необходимые комментарии.}

\subsubsection{В состав сопроводительной документации должны входить:
методические указания по работе с комплексом программ.}

\eject

\section{СТАДИИ И ЭТАПЫ РАЗРАБОТКИ}

Часть функциональности \vrio уже реализована:
\begin{itemize}
\item коммутатор \vrio (\verb#vrio_switch.py#);
\item блок работы с транзакциями MAINTENANCE для модели
контроллера RapidIO СБИС 1890ВМ8Я;
\item сценарии запуска двух эмуляторов qemu и коммутатора \vrio для
проверки функционирования транзакций MAINTENANCE.
\end{itemize}

На первом этапе требуется реализовать следующую функциональность \vrio:
\begin{itemize}
\item реализовать поддержку операций MESSAGE;
\item методику запуска системы из двух экземпляров qemu, коммутатора \vrio,
необходимых каналов \vrio и проверку работы команд PMON2000 \texttt{link}
и \texttt{riosend}.
\end{itemize}

На втором этапе потребуется доработка \vrio в части операций MESSAGE
для реализации поддержки передачи данных при помощи стека RapidIO для ОС Linux
разработки НИИСИ РАН. Задание для второго этапа не уточняется в данной версии
технических требований.

% \begin{center}
% \noindent\hfill Таблица 1. Стадии и этапы разработки
% \small
% \noindent\begin{tabular}{|c|p{44mm}|p{20mm}|p{35mm}|}
% \hline
% No. & Наименование & Срок исполнения & Результат \\
% \hline
% 1 & & & \\
% \hline
% 2 & & & \\
% \hline
% 3 & & & \\
% \hline
% \end{tabular}
% \end{center}

\section{ПРИМЕЧАНИЕ}
В процессе выполнения работы возможно изменение требований ТЗ установленным порядком.

\eject

\section*{ПРИЛОЖЕНИЕ: формат пакета сети \vrio}

Аналогично реальной коммуникационной среде RapidIO, пакеты виртуальной
сети \vrio разделены на поля, относящиеся к логическому,
транспортному и физическому уровню.

Поля физического и транспортного уровней формируют заголовок,
присущий всем пакетам \vrio.

Поля заголовка пакета \vrio и их размеры (в байтах):

\begin{verbatim}
+--------+--------+----------+------+-----+-------+---------------+
| format | transp | reserved | dest | src | lllen | logical layer |
|  type  |  type  |          |  ID  |  ID |       |     fields    |
+--------+--------+----------+------+-----+-------+---------------+
|   1    |   1    |    2     |  2   |  2  |   4   |     lllen     |
+--------+--------+----------+------+-----+-------+---------------+
\end{verbatim}

\begin{itemize}
\item format type --- тип пакета (1 байт). Краткое обозначение ftype;
\item transp type --- тип транспортного уровня (1 байт), по умолчанию здесь стоит 0. Краткое обозначение ttype; (Примечание: реальной системе 2-битное поле tt определяет тип используемой
адресации транспортного уровня (8- или 16-разрядные поля destination ID и source ID);
\item reserved --- зарезервированное поле (2 байта), по умолчанию 0;
\item dest ID --- адрес получателя пакета (2 байта). Краткое обозначение dId;
\item src ID --- адрес отправителя пакета (2 байта). Краткое обозначение sId;
\item lllen --- длина следующих полей логического уровня (4 байта);
\item logical layer fields --- поля логического уровня, их тип и количество
определяется типом операции на логическом уровне.
\end{itemize}

\eject

\subsection*{MAINTENANCE READ REQUEST}

\begin{verbatim}
+-------+-------+--------+-----+--------+-----+
|  hop  | ttype | rdsize | TID | offset | pad |
| count |       |        |     |        |     |
+-------+-------+--------+-----+--------+-----+
|   1   |   1   |   1    |  1  |   4    |  4  |
+-------+-------+--------+-----+--------+-----+
\end{verbatim}

\subsection*{MAINTENANCE READ RESPONSE}

\begin{verbatim}
+-------+-------+--------+-----+--------+-----+
|  hop  | ttype | status | TID |  data  | pad |
| count |       |        |     |        |     |
+-------+-------+--------+-----+--------+-----+
|   1   |   1   |   1    |  1  |   4    |  4  |
+-------+-------+--------+-----+--------+-----+
\end{verbatim}

\subsection*{MAINTENANCE WRITE REQUEST}

\begin{verbatim}
+-------+-------+--------+-----+--------+-----+
|  hop  | ttype | wrsize | TID | offset | data|
| count |       |        |     |        |     |
+-------+-------+--------+-----+--------+-----+
|   1   |   1   |   1    |  1  |   4    |  4  |
+-------+-------+--------+-----+--------+-----+
\end{verbatim}

\subsection*{MAINTENANCE WRITE RESPONSE}

\begin{verbatim}
+-------+-------+--------+-----+--------+-----+
|  hop  | ttype | status | TID |  pad   | pad |
| count |       |        |     |        |     |
+-------+-------+--------+-----+--------+-----+
|   1   |   1   |   1    |  1  |   4    |  4  |
+-------+-------+--------+-----+--------+-----+
\end{verbatim}

\eject

\subsection*{MESSAGE}

\begin{verbatim}
+-------+-------+ - - - - - +
| letter| mbox  |  D A T A  |
|       |       |           |
+-------+-------+ - - - - - +
|   1   |   1   |    H Z    |
+-------+-------+ - - - - - +
\end{verbatim}

\subsection*{RESPONSE}

\begin{verbatim}
+-------+--------+-------+-------+
| ttype | status | letter| mbox  |
|       |        |       |       |
+-------+--------+-------+-------+
|   1   |   1    |   1   |   1   |
+-------+--------+-------+-------+
\end{verbatim}

\end{document}

\endinput
